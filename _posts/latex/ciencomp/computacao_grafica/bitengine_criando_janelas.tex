\documentclass[a4paper,oneside,12pt]{article}
\usepackage[portuguese]{babel}
\usepackage{graphicx}
\usepackage{listings}
\usepackage{indentfirst}
\usepackage{fontspec}
\usepackage{hyperref}

\usepackage{xcolor}

\definecolor{codegreen}{rgb}{0,0.6,0}
\definecolor{codegray}{rgb}{0.5,0.5,0.5}
\definecolor{codepurple}{rgb}{0.58,0,0.82}
\definecolor{backcolour}{rgb}{0.95,0.95,0.92}
\definecolor{bg}{rgb}{0.8, 0.8, 0.8}

\lstdefinestyle{mystyle}{
    backgroundcolor=\color{backcolour},   
    commentstyle=\color{codegreen},
    keywordstyle=\color{magenta},
    numberstyle=\tiny\color{codegray},
    stringstyle=\color{codepurple},
    basicstyle=\ttfamily\footnotesize,
    breakatwhitespace=false,         
    breaklines=true,                 
    captionpos=b,                    
    keepspaces=true,                 
    numbers=left,                    
    numbersep=5pt,                  
    showspaces=false,                
    showstringspaces=false,
    showtabs=false,                  
    tabsize=2
}

\lstset{style=mystyle}

\newcommand{\mybox}[1]{
    \colorbox{backcolour}{\textbf{#1}}
}

\linespread{1.5}
\setmainfont{Times New Roman}

\title{bitEngine - Criando Janelas Multiplataformas}
\author{Canoi Gomes}
\date{23 de Abril de 2023}

% \newtcbox{\mybox}{on line,boxrule=0pt,boxsep=0pt,colback=lightgray,top=1pt,bottom=1pt,left=1pt,right=1pt,arc=0pt,fontupper=\ttfamily}

\begin{document}
\maketitle
\newpage

\tableofcontents
\newpage

\section{Introdução}

Dessa vez decidir começar um projeto sobre uma parte que venho querendo aprender a um tempo, que seria como criar o contexto básico pra um jogo (janela, input, gráficos e áudios) utilizando somente bibliotecas do próprio sistema, em resumo, eu queria entender mais como bibliotecas como SDL2 e GLFW funcionam por baixo dos panos.

Nisso (como sempre faço na minha vida) decidi criar um projeto pra focar nos estudos dessa parada, bite, a ideia é:

\begin{itemize}
    \item Conseguir criar uma janela pelo menos em desktop (Windows, Linux e Mac) e web (Emscripten)
    \item Criar um render básico utilizando OpenGL (OpenGLES2 com Emscripten), e carregar somente algumas funções específicas do modern OpenGL que vão ser necessárias (criar shaders, framebuffers, …).
    \item Tocar pelo menos 1 áudio (a ideia é fazer um mixer, mas vamo vê né).
    \item Filesystem básico (ler e escrever arquivos, listar diretórios, …).
\end{itemize}

\noindent
A ideia é ter algo como:
\begin{lstlisting}[language=C]
#include <bite.h>
#if defined(__EMSCRIPTEN__)
    #include <emscripten.h>
#endif

void main_loop(void* arg) {
    be_Context* ctx = (be_Context*)arg;
    bite_poll_events(ctx);
    // render suff
    bite_swap(ctx);
}

int main(int argc, char** argv) {
    be_Config conf = bite_init_config("Hello Window", 640, 380);
    be_Context* ctx = bite_create(&conf);
#if defined(__EMSCRIPTEN__)
    emscripten_set_main_loop_arg(main_loop, ctx, 0, 1);
#else
    while(!bite_should_close(ctx)) main_loop(ctx);
#endif
    bite_destroy(ctx);
    return 0;
}
\end{lstlisting}

Onde por trás vai ser criado o contexto específico pra cada plataforma:

\begin{lstlisting}[language=C]
#if defined(_WIN32)
	#include <windows.h>
#elif defined(__EMSCRIPTEN__)
	#include <emscripten.h>
	#include <emscripten/html5.h>
#else
	#include <X11/Xlib.h>
	#include <GL/glx.h>
#endif

be_Context* bite_create(const be_Config* conf) {
#if defined(_WIN32)
	// Win32 Window and WGL context creation
#elif defined(__EMSCRIPTEN__)
	// Emscripten context creation
#else
	// Linux Window and GLX context creation
	// other systems .....
#endif
}
\end{lstlisting}

Se estiverem interessados em como funciona a criação da janela pra cada plataforma, esse artigo dá uma pincelada legal no assunto: \url{https://zserge.com/posts/fenster/}

\section{Renderizando com OpenGL}

Na parte de renderização vai OpenGL mesmo, que como eu disse, funciona bem pro meu escopo (Desktop e Web). Pra isso preciso carregar um contexto OpenGL que suporte extensões, já que as libs padrão de cada plataforma (GLX no Windows e WGL no Windows) só nos dão um contexto com uma versão antiga (versão 1.4 se não me engano), e cada plataforma tem sua maneira de carregar um contexto mais moderno. No Windows, por exemplo, é necessário criar uma ``dummy window'' com um contexto antigo, somente pra ser capaz de carregar a função responsável por criar o contexto mais novo, depois disso ela é simplesmente deletada, no Linux não é necessário (outras plataformas provavelmente tem suas especificidades também, mas não cheguei lá ainda).

\href{https://gist.github.com/nickrolfe/1127313ed1dbf80254b614a721b3ee9c}{Exemplo no Windows}

\href{https://apoorvaj.io/creating-a-modern-opengl-context/}{Tutorial para Linux}

Tendo o “contexto moderno” carregado, ainda é preciso carregar as funções que eu vou utilizar, e pra isso existe a função \mybox{GetProcAddress} de cada lib (\mybox{glXGetProcAddress} no Linux ou \mybox{wglGetProcAddress} no Windows).

\begin{lstlisting}
typedef GLuint glCreateProgramProc(void);

static glCreateProgramProc* glCreateProgram = 0;

#if defined(_WIN32)
    #define biteGetProcAddress wglGetProcAddress
#elif defined(__linux__)
    #define biteGetProcAdress glXGetProcAddress
#else
    #define biteGetProcAddress(x) ((void)(x))
#endif

int init_opengl_procs(void) {
    glCreateProgram = (glCreateProgramProc*)biteGetProcAdress("glCreateProgram");
    return 0;
}
\end{lstlisting}

Vale a pena dar uma olhada em outros loaders como o \href{https://glad.dav1d.de/}{glad} e o \href{https://glew.sourceforge.net/}{GLEW}.

Tem uma lib minha que faz algo parecido com o que eu quero fazer aqui, \href{https://github.com/cafe-engine/tea}{tea}, que é basicamente carregar somente o mínimo de funções necessárias e criar abstrações em cima delas.

Pra ter o básico pra suportar shaders, por exemplo, seriam necessárias:

\noindent\rule{\textwidth}{0.4pt}
\begin{itemize}
    \item glCreateShader
    \item glShaderSource
    \item glCompileShader
    \item glGetShaderiv
    \item glGetShaderInfoLog
    \item glDeleteShader
\end{itemize}
\noindent\rule{\textwidth}{0.4pt}
\begin{itemize}
    \item glCreateProgram
    \item glAttachShader
    \item glLinkProgram
    \item glGetProgramiv
    \item glGetProgramInfoLog
    \item glDeleteProgram
    \item glUseProgram
\end{itemize}
\noindent\rule{\textwidth}{0.4pt}

\noindent
Sem expor isso pro usuário, mas sim abstraindo o processo em outras funções:
\begin{lstlisting}[language=C]
be_Shader* bite_create_shader(const char* vert_src, const char* frag_src) {
    be_Shader* shader = NULL;
    GLuint program;
    GLuint vert, frag;

    vert = glCreateShader(GL_VERTEX_SHADER);
    glShaderSource(vert_src);
    // ....
    
    program = glCreateProgram();
    glAttachShader(program, vert);
    glAttachShader(program, frag);
    // ...

    shader->handle = program;
    glDeleteShader(vert);
    glDeleteShader(frag);

    return shader;
}
\end{lstlisting}

E é isto.

\section{Considerações}

Fora a renderização também vão ter outros pontos pra se lidar, como por exemplo:

\begin{itemize}
    \item Eventos (janela movendo/redimensionando, tecla pressionada, …), que na real é bem tranquilo, o mais chatinho é lidar com a questão multiplataforma da parada mesmo, já que no caso de teclas pressionadas os KeyCodes são diferentes, por exemplo, então tu vai ter que criar os seus próprios e filtrar por plataforma pra retornar pro usuário o certo.
    \item Áudio, que sinceramente ainda é um mistério pra mim.
\end{itemize}

Eu pretendo (ou pelo menos espero conseguir) postar devlogs a medida que for aprendendo sobre os assuntos.

E outra coisa que to pensando em fazer é separar os backends em arquivos .c diferentes, queria muito ter um único \mybox{.h} e \mybox{.c} pra facilitar portabilidade, mas é horrível de mexer com tanto \mybox{\#ifdef}.

\end{document}