\documentclass[a4paper,oneside,12pt]{article}
\usepackage[portuguese]{babel}
\usepackage{graphicx}
\usepackage{listings}
\usepackage{indentfirst}
\usepackage{fontspec}
\usepackage{hyperref}

\usepackage{xcolor}

\definecolor{codegreen}{rgb}{0,0.6,0}
\definecolor{codegray}{rgb}{0.5,0.5,0.5}
\definecolor{codepurple}{rgb}{0.58,0,0.82}
\definecolor{backcolour}{rgb}{0.95,0.95,0.92}
\definecolor{bg}{rgb}{0.8, 0.8, 0.8}

\lstdefinestyle{mystyle}{
    backgroundcolor=\color{backcolour},   
    commentstyle=\color{codegreen},
    keywordstyle=\color{magenta},
    numberstyle=\tiny\color{codegray},
    stringstyle=\color{codepurple},
    basicstyle=\ttfamily\footnotesize,
    breakatwhitespace=false,         
    breaklines=true,                 
    captionpos=b,                    
    keepspaces=true,                 
    numbers=left,                    
    numbersep=5pt,                  
    showspaces=false,                
    showstringspaces=false,
    showtabs=false,                  
    tabsize=2
}

\lstset{style=mystyle}

\newcommand{\mybox}[1]{
    \colorbox{backcolour}{\textbf{#1}}
}

\linespread{1.5}
\setmainfont{Times New Roman}

\usepackage{dirtree}

\title{Selene Engine}
\author{Canoi Gomes}
\date{19 de Outubro de 2023}

\begin{document}

\maketitle
\newpage

\tableofcontents
\newpage

\section{Introdução}

Nas ultimas semanas estava focado em reestruturar um dos meus projetos que é basicamente uma game framework/engine que utilize Lua como linguagem de script, o projeto inicialmente nasceu como \href{https://github.com/canoi12/poti}{poti}, mas há algum tempo estava tentando ressignificar outro dos meus projetos, a \href{https://github.com/canoi12/selene}{selene}, e por achar que o nome casa mais com o projeto em questão, decidi fazer essa troca.

Inicialmente eu estava focado em fazer a framework base em C e operar ela utilizando Lua, então a ideia seria ter um renderizador básico em C, uma engine de áudio básica também, etc. Maaas, decidi seguir por um caminho diferente, e até explorar mais a ideia inicial do projeto que é a de ter um core simples e o projeto ser o mais modular possível via Lua, então ao invés de construir essas estruturas em C usando as libs (SDL2, OpenGL, …), achei melhor expor as funções das lib pra Lua e construir a framework lá.

Então por exemplo, uma função de desenhar um retângulo em C:

\begin{lstlisting}[language=C]
static int l_graphics_fill_rectangle(lua_State* L) {
    float *c = RENDER()->color;
    float x, y, w, h;
    x = luaL_checknumber(L, 1);
    y = luaL_checknumber(L, 2);
    w = luaL_checknumber(L, 3);
    h = luaL_checknumber(L, 4);
    set_texture(RENDER()->white_texture);
    float vertices[] = {
        x, y, c[0], c[1], c[2], c[3], 0.f, 0.f,
        x + w, y, c[0], c[1], c[2], c[3], 1.f, 0.f,
        x + w, y + h, c[0], c[1], c[2], c[3], 1.f, 1.f,

        x, y, c[0], c[1], c[2], c[3], 0.f, 0.f,
        x, y + h, c[0], c[1], c[2], c[3], 0.f, 1.f,
        x + w, y + h, c[0], c[1], c[2], c[3], 1.f, 1.f,

    };

    VertexFormat* v = (VertexFormat*)vertices;
    push_vertices(VERTEX(), 6, v);
}
\end{lstlisting}

\noindent
Vira isso:

\begin{lstlisting}[language={[5.2]Lua}]
function graphics.fill_rectangle(x, y, width, height)
	set_image()
	set_draw_mode('triangles')

	local r,g,b,a = table.unpack(current.draw_color)
	local vertex_data = default.batch.data
	default.batch:push(x, y, r, g, b, a, 0.0, 0.0)
	default.batch:push(x+w, y, r, g, b, a, 0.0, 0.0)
	default.batch:push(x+width, y+height, r, g, b, a, 0.0, 0.0)

	default.batch:push(x, y, r, g, b, a, 0.0, 0.0)
	default.batch:push(x+width, y+height, r, g, b, a, 0.0, 0.0)
	default.batch:push(x, y+height, r, g, b, a, 0.0, 0.0)
end
\end{lstlisting}

Obviamente isso tem custos em relação a performance, mas dependendo da proposta do projeto funciona muito bem, fora que a ideia é a partir desse ponto construir uma engine em cima disso, então posso usar toda a modularidade que Lua me permite pra otimizar boa parte dessas chamadas de funções, e caso necessário é só refazer partes em C.

Perceba aqui a mudança na estrutura da biblioteca:

\begin{tabular}{|p{.3\linewidth}|p{.4\linewidth}|p{.3\linewidth}|}
    \hline
    \mybox{poti (old)} & \mybox{selene (C)}& \mybox{selene (Lua)}\\
    \hline
    \dirtree{%
        .1 poti.
        .2 audio.
        .3 AudioData.
        .2 event.
        .2 filesystem.
        .3 File.
        .2 gamepad.
        .3 Gamepad.
        .2 graphics.
        .3 Texture.
        .2 joystick.
        .2 keyboard.
        .2 mouse.
        .2 window.
    }&
    \dirtree{%
        .1 selene (C).
        .2 Data.
        .2 audio.
        .3 Decoder.
        .2 font.
        .2 fs.
        .3 File.
        .2 gl.
        .3 Buffer.
        .3 Framebuffer.
        .3 Program.
        .3 Shader.
        .3 Texture.
        .3 VertexArray.
        .2 image
        .2 linmath
        .3 Mat4.
        .2 sdl2.
        .3 AudioDeviceID.
        .3 AudioStream.
        .3 Event.
        .3 Gamepad.
        .3 GLContext.
        .3 Joystick.
        .3 Window.
    }&
    \dirtree{%
        .1 Selene (Lua).
        .2 audio.
        .3 Music.
        .3 Sound.
        .2 filesystem.
        .2 graphics.
        .3 Batch.
        .3 Canvas.
        .3 Font.
        .3 Image.
        .3 Rect.
        .3 Shader.
        .2 gamepad.
        .2 joystick.
        .2 keyboard.
        .2 mouse.
    }\\
    \hline
\end{tabular}
\break

Como expliquei acima, agora estou focando muito mais em expor as minhas libs para Lua.

O executável ainda terá um pequeno script de boot embutido nele exatamente para dizer como nossa aplicação irá iniciar. A lógica por trás dele é de adicionar a pasta do executável no path do Lua, e então buscar por um módulo \mybox{main}.

\begin{lstlisting}[language={[5.2]Lua}]
local sdl = selene.sdl2
local function add_path(path)
    package.path = path .. '?.lua;' .. path .. '?/init.lua;' .. package.path
end
return function(args)
    add_path(sdl.GetBasepath())
    if selene.args[2] then add_path(selene.args[2]) end
    return require('main')
end
\end{lstlisting}

No módulo main você poderá fazer uso dos scripts da framework em Lua, que estarão na pasta \mybox{core/} que será distribuído junto com o executável.

A partir daí me aproveitar da modularidade do Lua pra criar um programa que possa ao mesmo tempo ter uma engine e na hora de distribuir ter só um runner. Então posteriormente distribuir também uma pasta \mybox{engine/}, \mybox{runner/}, \mybox{editor/}, e assim vai. Fazer uma API para plugins também.

Enfim, no final a decisão foi mais pra não me preocupar tanto com performance e me aproveitar mais da liberdade do Lua pra criar diferentes sistemas e um certo nível de modularidade, como mudar o renderizador em tempo real (que é uma das ideias, inclusive).



\end{document}