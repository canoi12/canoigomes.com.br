\documentclass[a4paper,oneside,12pt]{article}
\usepackage{fontspec}
\usepackage{indentfirst}

\usepackage{datetime}
\usepackage[portuguese]{babel}
\usepackage[margin=1in]{geometry}
\usepackage{graphicx}

\linespread{1.5}

\setmainfont{Times New Roman}

\title{Toda Ação Mira um Bem}
\author{Canoi Gomes}

\begin{document}

\begin{center}
    \begin{center}
        \includegraphics[width=.2\linewidth]{../brasao_flat.png}
    \end{center}

    \begin{center}
        UNIVERSIDADE FEDERAL DO RIO GRANDE DO NORTE \\
        CENTRO DE CIÊNCIAS HUMANAS, LETRAS E ARTES \\
        DEPARTAMENTO DE FILOSOFIA \\
        BACHARELADO EM FILOSOFIA
    \end{center}

    \vspace{2cm}

    Ética a Nicômaco, Livro I \\
    \textbf{\Large{Por Que Toda Ação Mira um Bem?}}

    \vspace{4cm}

    \begin{center}
        \large{Canoi Gomes de Aguiar}
    \end{center}

    \vspace{6cm}

    NATAL - RN \\
    Abril de 2024
\end{center}
\newpage

Logo no início de Ética a Nicômaco, Aristóteles nos faz a afirmação de que ``Toda ação mira um bem qualquer'', mas primeiro precisamos entender os conceitos de ``ação'' e ``bem'' para entender a frase. Falando em termos práticos, realizamos toda ação em favor de uma finalidade, e essas finalidades são muitas, na ação de correr, por exemplo, podemos estar treinando para uma competição, por motivos de saúde, ou simplesmente por gostar de correr, é importante entender que ``ações'' também podem ser entendidas como ``arte'', ``técnica'' ou ``atividade'', Aristóteles nos dá um exemplo quando diz que ``o fim da arte médica é a saúde, o da construção naval é um navio, o da estratégia é a vitória''. Então o filósofo nos fala de dois tipos de ações, as que as finalidades se encerram na própria atividade, chamadas de \textbf{Práxis}, e aquelas na qual é produzido algum bem externo como finalidade, as chamadas \textbf{Poiesis}. Voltando para o exemplo anterior, poderíamos dizer que correr é uma práxis, já que seu fim não produz nenhum bem concreto e se encerra na própria atividade, já um artesão que produz tênis para corredores estaria praticando uma poiesis.

Só que esses bens produzidos pelas poiesis podem ter diferentes finalidades, e de mesmo modo as práxis não necessariamente tem uma única finalidade, como vimos na atividade de correr, mas Aristóteles diz que essas finalidades vão mirar a de alguma outra arte fundamental, por exemplo, tanto o cuidador de cavalos quanto o artesão que fabrica os acessórios para cavalos tem como finalidade a arte da equitação, e essa última vai ter a finalidade na sua própria excelência, do contrário essa busca seria infinita, esse bem mais absoluto Aristóteles chama de \textbf{sumo bem}, pois ele só se subordina a si mesmo, e esse bem maior parece ser a \textbf{felicidade} (ou \textbf{Eudaimonia}), pois todas as nossas ações que sejam mirando o prazer, a honra ou a justiça, parecem sempre mirá-la também, já a felicidade não miramos tendo em vista nenhuma outra coisa se não a própria felicidade. Aqui Aristóteles começa a se afastar dos platônicos ao dizer que não existe um único bem, afinal os bens das atividades são muitos, e alguns destes vão se subordinar às de outras artes a fim de buscar o sumo bem/felicidade, porém a felicidade é relativa ao indivíduo, para alguns será ela os prazeres, para outros a riqueza, dentre muitos outros, mas os bens que nos interessam são os que agradam a nossa \textbf{alma}.

Para Aristóteles, não só buscamos a eudaimonia, como essa busca também é a finalidade máxima da nossa vida, da nossa alma, não no sentido espiritual e sim de nossa vida ativa enquanto seres humanos, desde o nascimento até a morte, e durante a vida a maioria dos homens parecem identificar a felicidade a partir dos prazeres, mas não são todos os prazeres que buscamos, pois quanto aos prazeres do corpo os homens parecem discordar sobre o que é aprazível, então estes seriam naturais, e os que procuramos são os da alma, que vão partir de um \textbf{princípio racional} (ou \textbf{Logos}), e igualmente precisamos usar a razão para identificá-los. Aristóteles nos fala de três tipos de vida, a dos \textbf{prazeres}, a \textbf{contemplativa} e a \textbf{política}, ele diz que devemos desconsiderar esta primeira, pois além de seus bens parecerem momentâneos, os que seguem essa vida parecem não usar a razão para fazer suas escolhas, a felicidade que estamos buscando é racional e duradoura, a vida contemplativa ele diz que deixemos para depois, e nos resta então a vida política. Se a vida política é a que devemos buscar, então igualmente a atividade que nos levaria à felicidade seria a prática da \textbf{ciência política}, pois ela que vai nos permitir definir quais são as ciências que os cidadãos de um Estado devem estudar, bem como outras artes dentro da \textbf{Polis} também parecem se submeter a ela, ele também fala que se a busca pela eudaimonia é a finalidade máxima de nossa vida, então a prática política seria igualmente a atividade máxima da vida humana.

A Polis grega foi criada exatamente para que os homens pudessem exercer a política a fim de legislar e buscar a felicidade máxima para os seus participantes, mas se a felicidade é relativa ao sujeito, então os legisladores devem ser capazes de fazer as escolhas certas a fim de trazer o bem-estar a todos os cidadãos. Quando falamos dos prazeres da alma, precisamos focar nos indivíduos que identificam felicidade com a sua honra, que são justos, que parecem ser assertivos em suas escolhas e buscam excelência nas suas ações, parecem ter um espécie de \textbf{sabedoria prática} e são indivíduos que vão se destacar na Polis por sua \textbf{virtude}, são as escolhas desses virtuosos que devemos observar. Aristóteles diz que desde que nossa capacidade para a virtude não tenha sido mutilada, nós podemos aprendê-la, mas que não é praticar uma única ação virtuosa que te torna virtuoso, é preciso criar um hábito que só vai se desenvolver à medida que sentirmos prazer pela virtude, pois a cada homem é agradável aquilo que ele ama, como atos justos para o amante da justiça, e atos virtuosos para o amante da virtude.

Então se os bens das ações são vários, e se todos aqueles bens que não tem fim em si mesmos irão se subordinar a algum outro mais absoluto, e o mais absoluto de todos será a felicidade, podemos dizer que toda ação mira um bem, pois todas nossas ações buscam a felicidade, sejam pelos prazeres, sejam pelas honras, na própria atividade ou em outra mais fundamental a esta, mas que se buscamos uma felicidade duradoura, então precisamos procurar a excelência de nossas ações, precisamos ser virtuosos dentro da nossa comunidade, mas não basta agir virtuosamente uma única vez, é necessária uma prática constante dessas ações até o fim de nossa vida.

\end{document}