\documentclass[a4paper,oneside,12pt]{article}
\usepackage[portuguese]{babel}
\usepackage{graphicx}
\usepackage{listings}
\usepackage{indentfirst}
\usepackage{fontspec}
\usepackage{hyperref}

\usepackage{xcolor}

\definecolor{codegreen}{rgb}{0,0.6,0}
\definecolor{codegray}{rgb}{0.5,0.5,0.5}
\definecolor{codepurple}{rgb}{0.58,0,0.82}
\definecolor{backcolour}{rgb}{0.95,0.95,0.92}
\definecolor{bg}{rgb}{0.8, 0.8, 0.8}

\lstdefinestyle{mystyle}{
    backgroundcolor=\color{backcolour},   
    commentstyle=\color{codegreen},
    keywordstyle=\color{magenta},
    numberstyle=\tiny\color{codegray},
    stringstyle=\color{codepurple},
    basicstyle=\ttfamily\footnotesize,
    breakatwhitespace=false,         
    breaklines=true,                 
    captionpos=b,                    
    keepspaces=true,                 
    numbers=left,                    
    numbersep=5pt,                  
    showspaces=false,                
    showstringspaces=false,
    showtabs=false,                  
    tabsize=2
}

\lstset{style=mystyle}

\newcommand{\mybox}[1]{
    \colorbox{backcolour}{\textbf{#1}}
}

\linespread{1.5}
\setmainfont{Times New Roman}

\title{A Terra Como Consciência Viva}
\author{Canoi Gomes}
\date{\today}

\begin{document}

\maketitle
\newpage

\tableofcontents
\newpage

\section{Introdução}

Durante esse período de 2 meses que fiquei parado durante a greve (pelo menos das atividades da UFRN),
tive um tempo para tentar ler algumas coisas.
Decidi dar uma chance ao Krenak, já tinha ouvido falar antes e durante as aulas voltei a voltar o pessoal comentando, e um dos meus objetivos na filosofia é exatamente tentar ir atrás desse conhecimento mais "alternativo" (entenda-se por pensamento que não necessariamente está na academia).

Nesse texto quero discutir mais a parte metafísica da ideia, pois foi algo que me veio a cabeça que me veio com as aulas dessas últimas semanas e na hora liguei com Krenak. Que é a ideia da Terra como tendo consciência e sendo um organismo vivo. Não no sentido de uma "mãe natureza", um espirito que está conosco no planeta, e sim o próprio planeta ter uma certa forma de consciência e conversar conosco de diversas maneiras. E isso me remeteu às ideias do Krenak, quando ele fala sobre como seu povo conversa com as montanhas, com os rios, e tem essa conexão mais próxima com a natureza.

E se esse planeta consegue se manifestar de diferentes formas? Podemos sentir sua respiração, seu choro, sentir sua raiva, mas temos dificuldade de interpretá-la. Já ela não só nos sente, como consegue nos ouvir e nos responder.

\section{Consciência}

\end{document}